% Last updated on 16 Nov 2006
% LaTeX resume using res.sty
% MODIFIED TO PUT NAME AND ADDRESS ON ALL PAGES
% \documentstyle[line,fancyheadings]{res}
\documentclass[line, 11pt]{res}
\usepackage{fancyheadings}
\usepackage{hyperref}
\usepackage{bchart}

\hypersetup{
    colorlinks=true,
    linkcolor=blue,
    filecolor=magenta,      
    urlcolor=blue,
}

%\lhead{\opening}                % put \opening (name) at top of every page
\rhead{}                        % right head empty
\cfoot{}                        % the footer is empty
\pagestyle{fancy}               % set pagestyle for the document

\setlength{\headheight}{-0.3in}   % make room for header
\setlength{\headrulewidth}{0pt} % suppress line drawn by default with fancyhead
\setlength{\textheight}{9.8in}  % reduce textheight because of header
% Redefine resume environment so it doesn't start with \opening
\renewenvironment{resume}{\begingroup}{\endgroup}
\renewcommand{\namefont}{\LARGE\emph}

\newcommand{\skillTScale}{0.9}
\newcommand{\skillTMaxVal}{10}
\newcommand{\skillTGap}{3pt}


%\definecolor{mycolor1}{rgb}{0.396, 0.6, 1}
%\definecolor{mycolor2}{rgb}{0.514, 1, 0.965}
%\definecolor{mycolor3}{rgb}{0.635, 1, 0.706}
%\definecolor{mycolor4}{rgb}{0.886, 1, 0.757}
%\definecolor{mycolor5}{rgb}{1, 0.965, 0.878}
%\definecolor{mycolor6}{rgb}{1,1,1}

%\definecolor{mycolor1}{RGB}{179, 236, 255}
%\definecolor{mycolor2}{RGB}{179, 186, 255}
%\definecolor{mycolor3}{RGB}{221, 179, 255}
%\definecolor{mycolor4}{RGB}{255, 179, 239}
%\definecolor{mycolor5}{RGB}{255, 179, 189}
%\definecolor{mycolor6}{RGB}{255, 218, 179}

\definecolor{mycolor1}{RGB}{246, 170, 161}
\definecolor{mycolor2}{RGB}{246, 237, 159}
\definecolor{mycolor3}{RGB}{185, 247, 156}
\definecolor{mycolor4}{RGB}{171, 249, 207}
\definecolor{mycolor5}{RGB}{174, 231, 249}
\definecolor{mycolor6}{RGB}{179, 178, 250}






\begin{document}

{\huge Nagarajan} % \hspace{10pt} {\large - Senior Software Engineer}

% \name{Nagarajan - Full stack / Frontend Senior Engineer}
% \address used twice to have two lines of address
%\address{10/231 DDA Flats, Madangir, New Delhi, India - 110062}
% \address{\href{mailto:nag.rajan@gmail.com}{nag.rajan@gmail.com}, +1 650 704 6280}
% \address{\href{https://www.motleytech.net}{https://www.motleytech.net}}

\begin{resume}
% \opening

\vspace{5pt}
\hrule
\vspace{-5pt}
I have deep interest in programming and mathematics, and I love solving challenging problems \& delighting clients.  I have 15+ years of solid technical experience spanning frontend, backend, algorithms and graphics.

I have technically lead fairly large projects - eg. lead the tech effort to build Lyft's customer support web portal \href{https://help.lyft.com}{https://help.lyft.com}, and lead the tech effort to build the web 3D graphics frontend for Ciena's Network Management system.

I also write a blog - \href{https://www.motleytech.net}{https://www.motleytech.net} - where I have a variety of technology and math oriented articles.

I am looking for a Frontend/Fullstack Tech Lead / IC role in an impactful project.

\vspace{8pt}
\hrule

\vspace{-7pt}
\section{Skill overview}
    \begin{itemize}
        \item \textbf{Frontend: } \\
        NEXT{\tiny.JS}, React, EmberJS, TypeScript, JS, HTML, CSS, ThreeJS, OpenGL
        \item \textbf{Backend: } \\
        NodeJS, Express, Python, Flask, Django, SQL, MongoDB, No-SQL
        \item \textbf{Others: } \\
        Contentful, Networking, HTTP, Rest, GraphQL, Grafana, Kibana, A11Y, Cypress, Storybook
    \end{itemize}

\vspace{5pt}
\hrule

\vspace{-7pt}

\section{Work Experience}
    \begin{itemize}
        \item{\sl Senior Software Engineer} \hfill January 2023 - Current\\
        Peloton Interactive, Santa Clara, California

        \item{\sl Lead Engineer} \hfill July 27, 2020 - January 2023\\
        Lyft, SF, California

        % \item{\sl Senior Software Engineer} \hfill April 27, 2020 - July 24, 2020\\
        % Thousand Eyes, SF, California
    
        \item{\sl Senior Software Engineer} \hfill February 16, 2010 - March 27, 2020\\
        Ciena / Cyan (Ciena acquired Cyan in Aug 2015), SF, California

        % \item{\sl Research Assistant} \hfill July 2007 - August 2009\\
        % Department of Mathematics, Stanford University

        \item{\sl Software Design Engineer} \hfill July 01, 2002 - June 06, 2007\\
        Texas Instruments, Bangalore, India\\
    \end{itemize}

\hrule

\vspace{-7pt}
\section{Educational Background}
    \begin{itemize}
        \item \textbf{M.S. - Stanford University} \\
        Computational and Mathematical Engineering
        \item \textbf{B.Engg. - Delhi Institute of Technology} \\
        Instrumentation and Control Engineering\\
    \end{itemize}

\vspace{-5pt}
\hrule

\vspace{-7pt}
\section{Peloton projects}
    \begin{itemize}
        \item \textbf{Peloton's Web Frontend}\\
As tech lead on the e-commerce frontend team, I was responsible for Peloton's web frontend and its customer acquisition funnel. The team executed a number of projects related to new customer acquisition - implemented complete site redesigns incrementally, improved customer purchase flows, integrated google shopping, and improved site reliability. In all these projects, we collaborated with the backend, content, marketing and other teams to deliver timely solutions. These features helped to improve user engagement and new user signups. Our e-commerce frontend team won Peloton's internal award for the best team - voted on by Product managers and other customers. 
    \end{itemize}
    
\vspace{-12pt}
\section{Lyft projects}
    \begin{itemize}
        \item \textbf{help.lyft.com}\\
Tech lead for a team of 4 which created Lyft's help web portal - \href{https://help.lyft.com}{https://help.lyft.com} - on Lyft's internal tech infrastructure (previously hosted on Zendesk). It was built using NEXT{\tiny.JS} (React, TypeScript), Python and Contentful. I designed, specced \& implemented the solution and worked with cross-functional teams to deliver the product in a timely manner.\\ \\ We upgraded the customer portal with automations and accessibility features, and then further built up the product to create a support platform. This platform enabled other teams within Lyft to build their own support portals on the platform. I earned a merit bonus for ownership and execution on this project.
    \end{itemize}

% ß\vspace{-12pt}
\section{Ciena/Cyan Projects}
    \begin{itemize}
        \item \textbf{3D Network Map}\\
In a team of two, we developed a 3D Graphical front end for the network management client. Working in Python and OpenGL, I developed higher level abstraction layers (graphical objects for the 3D viewer) over OpenGL primitives. I also developed tools for multi layered visualization of network services. Later, I was the lead developer for the project to port this work to a web browser based front-end using EmberJS and ThreeJS.

        \item \textbf{Trail Analyzer}\\
Network service providers had long requested better tools to efficiently debug network problems (they still do so). To solve a frequently faced issue of discovering root cause on disrupted services, I developed a tool - Trail Analyzer - to visualize a network service from end to end, while displaying live status information from all network devices that it passes through and highlighting any issues on each of those. This tool was highly appreciated by the customers and saved them significant time in debugging problems.

        \item \textbf{Large graph visualization}\\
As networks grew larger, it became ineffective to visualize the whole network all at once. I developed a novel algorithm to effectively visualize large sparsely connected graphs. I also developed a client-server protocol to efficiently transmit relevant parts of the network graph, and updates, between the back-end and the UI. I filed a patent for \href{https://patentscope.wipo.int/search/en/detail.jsf?docId=WO2015041751&tab=PCTBIBLIO}{Network Visualization System and Method} based on this work.

        \item \textbf{Hiring screening tool}\\
    I created a web application for the hiring team at Cyan to screen prospective employees. Using this app, the hiring team could email homework problems to applicants. The recipient could then work on the problem on their own time and submit the solutions into the app (online code editor for C and Python). The app would run linting checks and automated tests to rank and rate the submission, and send emails to notify the hiring team of submissions. The project was made using Python, Flask, SQL and Javascript.

    \end{itemize}

\vspace{-10pt}
\section{Stanford University Projects}
    \begin{itemize}
        \item \textbf{Cloth Simulation}\\
        Wrote a cloth simulation framework in C++, which solves a partial
        differential equation based on the physics of the cloth ( Baraff and
        Witkin, 1998, \emph{Large Steps in cloth simulation} ).
        I implemented and implicit solver using the  conjugate gradient iterative algorithm
        and also took friction into account while interacting
        with objects in the environment.

        \item \textbf{Optimization - Brachistochrone}\\
        Implemented a numerical optimization tool in Matlab and used it to
        solve the brachistochrone problem (the path of least time for a
        particle under the effect of gravity). The tool implemented the
        linesearch using gamma conditions and BFGS based direction.
    
    \end{itemize}

\vspace{-10pt}
\section{Texas Instruments Projects}
    \begin{itemize}
        \item \textbf{Pocket Imager Handheld Projector - Texas Instruments / Samsung}\\
        Joint effort between TI and Samsung, the DLP technology based Samsung
        Pocket Imager (P300 model) was the smallest and world's first LED
        based projector. I implemented new modules for controlling the LED
        driver, characterizing LED output color and dynamic feedback control 
        of White Point. I then closely assisted Samsung onsite in South Korea to solve 
        critical last minute issues and bring the product to market in time. Received the \textbf{DLP Silver 
        Olympian} award for execution on the project - awarded to top 3\% of DLP employees worldwide.

        \item \textbf{Business Projector - DLP Group, Texas Instruments}\\
        Responsible for Image improvement algorithms, drivers and 
        the color wheel motor control algorithm. Designed and implemented driver 
        module for new color wheel motor driver, feedback speed control algorithm 
        and the Dynamic Black algorithm. 
        
        \item \textbf{DSL modem firmware - ADSL Group, Texas Instruments}\\
        Our team modified an existing TI DSL implementation to work in the 
        Japanese ISDN noise environment and to interoperate with existing third 
        party solutions. I implemented the Handshake, Power Spectrum Management 
        and Dynamic rate adaptation modules over a 2 year period.
    \end{itemize}


% \vspace{-10pt}
% \section{Previous Projects}
%     \begin{itemize}
%         \item \textbf{Design of an autonomous Hexapod robot}\\
%         As the undergraduate final year project, we designed a Hexapod - six
%         legged robot - with autonomous motion capability. The robot used
%         ultrasonic and infrared sensors to sense its surroundings, could avoid
%         obstacles and was also programmed to escape from simple mazes.\\

%         %\item \textbf{Optimizer for C language}\\
%         %Developed a parser analyzer for C programs which could also suggest optimizations, indent the code and remove recursion from a recursive function. It could also implement naming conventions. This project won the first prize in the fortnight programming contest at NSIT.
    
%         %\item \textbf{LDPC Encoder and Decoder}\\
%         %Implemented encoder and soft and hard Decoders for LDPC codes. Also implemented graph algorithms to find loops of minimum length in the parity check matrix, which gives the error resilience of the codes.
%     \end{itemize}

\vspace{8pt}
\hrule
% \vspace{-5pt}

% \section{Awards}
%     \begin{enumerate}
%         \item 2006 : Won the TI DLP Silver Olympian award for execution on the Samsung Pocket Imager.
%         \item 2001 : First prize in DITECH Samadhan, the fortnight programming contest organized by NSIT, sponsored and judged by Hughes Software Systems. Project - Optimizer for C language.
%         \item 2001 : Second prize in DITECH Overnight programming contest organized by NSIT.
%     \end{enumerate}

\vspace{-3pt}
\section{Contact}
    \vspace{5pt}
    \begin{tabular}{ll}
    \textbf{Phone} & +1 650 704 6280\\
    \textbf{Email address} & \href{mailto:nag.rajan@gmail.com}{nag.rajan@gmail.com}\\
    \textbf{Website} & \href{https://www.motleytech.net}{https://www.motleytech.net}\\
    \textbf{Visa Status} & H1B sponsorship required, I-140 already approved\\
    \end{tabular}

% \address{\href{mailto:nag.rajan@gmail.com}{nag.rajan@gmail.com}, +1 650 704 6280}
% \address{\href{https://www.motleytech.net}{https://www.motleytech.net}}

\end{resume}

\end{document}

